\documentclass[11pt,a5paper]{article}
\usepackage{cooltooltips}
\usepackage[T1]{fontenc}
\usepackage{bookman}
\usepackage[activate]{pdfcprot}
\usepackage{geometry}
\usepackage{parskip}
\usepackage{color}
\usepackage{titlesec}
\usepackage[pdfpagemode=UseNone]{hyperref}

% Define this document's metadata.
\title{\textcolor{titlecolor}{\bfseries Troubleshooting the \vlf}}
\author{Scott Pakin \\ \textit{scott+vfaq@pakin.org}}
\hypersetup{%
  pdftitle={Troubleshooting the Visual LaTeX FAQ},
  pdfauthor={Scott Pakin, scott+vfaq@pakin.org},
  pdfsubject={Helping to diagnose problems with interactive PDF features},
  pdfkeywords={Visual LaTeX FAQ, PDF, hyperlinks, widgets, Web browser}
}

% Define a helpful shortcut.
\newcommand*{\vlf}{\emph{Visual \LaTeX{} FAQ}}

% Introduce a little extra space between paragraphs.
\parskip=0.8\baselineskip
\advance\parskip by 0pt plus 2pt

% Define a background color, text color, and hyperlink color.
\definecolor{bgcolor}{rgb}{0.95,0.92,1}
\definecolor{fgcolor}{rgb}{0,0.3,0}
\definecolor{titlecolor}{rgb}{0,0,1}
\hypersetup{%
  menubordercolor={0 0 1},
  linkbordercolor={0 0 1},
  urlbordercolor={0 0 1}
}

% Change the formatting of section titles.
\titleformat{\section}{\Large\bfseries\color{titlecolor}}{\thesection.}{1em}{}
\newcommand{\sectionbreak}{\clearpage}

%%%%%%%%%%%%%%%%%%%%%%%%%%%%%%%%%%%%%%%%%%%%%%%%%%%%%%%%%%%%%%%%%%%%%%%%%%%

\begin{document}
\pagecolor{bgcolor}
\hypertarget{link:title}{\maketitle}
\color{fgcolor}
\sloppy

The \vlf{} makes extensive use of PDF hyperlinks, widgets, and
form fields.  Not all of these are supported by every PDF
viewer.  Furthermore, even PDF viewers that support hyperlinks
may need to be configured to launch a Web browser when a link to a Web
page is activated.

The purpose of this document is to help diagnose problems involving the
interactive features of the \vlf.

Ready?  Then let's begin\dots


\section{Internal hyperlinks}

The simplest form of hyperlink jumps from one page of a document to
another page in the same document:

\begin{center}
  \hyperlink{link:title}{Return to page~1.\strut}
\end{center}

The text ``Return to page~1'' is a hyperlink.  Activating it should
take you back to the first page of this document.

If nothing happens then your PDF viewer probably doesn't
support any sort of hyperlink.  You will get little use out of the
\vlf{} unless you switch to a PDF viewer that supports
hyperlinks.

Let's move on to the next test\dots


\section{External hyperlinks}

The PDF format supports hyperlinks to other documents, including
non-PDF documents.  All of the hyperlinks in the \vlf{} refer to Web
pages.  The following hyperlink points to the \TeX\ FAQ home page:

\begin{center}
  \href{https://texfaq.org/}{Open a Web browser.\strut}
\end{center}

If nothing happens when you activate ``Open a Web browser'' then one
of the following is likely to be the case:

\begin{itemize}
  \item Your PDF viewer doesn't support external hyperlinks.
  You will get little use out of the \vlf{} unless you switch to a
  PDF viewer that supports hyperlinks.

  \item You need to tell your PDF viewer which application you want to
  use as your Web browser.  In some of Adobe's PDF viewers, the
  browser-selection dialog can be found under
  \Acrobatmenu{GeneralPrefs}{Edit$\rightarrow$\linebreak[0]Preferences$\rightarrow$\linebreak[0]Internet\strut}.
  The procedure to associate a Web browser with external hyperlinks
  will of course be different in other PDF viewers.
\end{itemize}

If your PDF viewer opened a Web browser at the \TeX\ FAQ home page,
then you'll likely be able to use the \vlf.

There's just one final test to try to make sure\dots


\section{Pop-up notes}

Instead of relying upon simple hyperlinks, the \vlf{} works some PDF
magic to convey additional information via tooltips and pop-up notes.
Unfortunately, few PDF viewers support all of the requisite PDF
mechanisms: \textsf{Text} annotations, \textsf{Widget} annotations,
form fields, and JavaScript.  See if the following hyperlink works in
your PDF viewer:

\begin{center}
  \cooltooltip[0 0 1]{A pop-up note}{Additional information would
    normally appear here.}{https://texfaq.org/}{The TeX FAQ}{Fancy,
    \vlf-style hyperlink\strut}
\end{center}

If all goes well, moving the mouse pointer over the ``Fancy,
\vlf-style hyperlink'' text should display a tooltip that reads,
``\textsf{The TeX FAQ}'', and pop up a small window entitled
``\textsf{A pop-up note}'' and containing a line of text.

Whether or not the tooltip and pop-up note work in your PDF viewer,
activating the hyperlink should open a Web browser at the \TeX\ FAQ
home page.  If so, then \mbox{\emph{Congratulations!}}; the \vlf{} is
almost certainly usable with your PDF viewer.  If not, then
\cooltooltiptoggle{\fcolorbox{titlecolor}{bgcolor}{disabling pop-ups}}
(clicking again re-enables them) may solve the problem.

\end{document}
